\documentclass[12pt,a4paper,oneside]{report}
\usepackage[utf8]{inputenc}
\usepackage{amsmath}
\usepackage{amsfonts}
\usepackage{amssymb}
\author{Alexander Weitzmann}
\title{Versioning How To}
\begin{document}
\thispagestyle{empty}
\maketitle

\section*{Purpose}
The vespucci project is evolving quickly and its save format changes from time to time. In order to support older files, we\footnote{In fact it was written by Dominic Scheurer and refined by our lab team.} have written a version update framework. It can support new versions with a few simple steps, which are described in this how-to.

\section*{Overview}
Creating a new version consists of the following steps:\\
\begin{itemize}
	\item[1.] Specify the transformation for:
	\begin{itemize}
		\item the semantic model
		\item the diagram notation
	\end{itemize}
	\item[2.] Write a new version class.
	\item[3.] Update the field for the newest version in the version template class.
\end{itemize}
\medskip
Note that you update from the last version to your new version, as the updating works incremental. Old versions must not be updated. Changing previous versions will most likely break the version chain.\\
After following these steps you're done and older diagram files should be updated to your new version. 
\newpage

\section*{Transformation specification}
The transformation specification describes how to update from one version to another. For every new version a new transformation must be specified; if that is not the case a new version is not needed at all. The transformation is realized with QVT (Query/View/Transformation). It consists of the following two parts:\\

The \textbf{model transformation} updates the semantic model of the diagram. E.g. setting new relationships between semantic element or adding new ones.\\

The \textbf{notation transformation} updates the view components of the diagram. E.g. introducing a new (visual) description field for ensembles or changing the shape of attachments.\\

Existing transformations can be found in the folder 'transformation' in the 'de.tud.cs.st.vespucci.versioning' project. This is also where you should put your new QVT-transformations.

\section*{Writing a new version class}
New version classes are to be placed in the 'de.tud.cs.st.vespucci.versioning.versions' package. All new versions must extend the class 'VespucciVersionTemplate', which is also found in said package. This template defines all methods, which must be implemented. They essentially: 
\begin{itemize}
	\item point to the associated transformations
	\item point to the preceding version
	\item define the creation date, which is also the default identifier.
	\item the new namespace of the new vespucci version
\end{itemize}
\medskip
The last step is to update the field 'NEWEST\_VERSION' in the version template, that points to the newest version. You must set this to your newly created version.

\end{document}